\documentclass[../../Hardware_Guide]{subfiles}
% Hier müssen keine Packages geladen werden, es werden automatisch die von masterdoc geladen,
% sowie die Konfigurationen.
% Bei includegraphics nur Bildname (Bsp: Bild.png) eingeben, da er in den angegebenen Pfade die Bilder sucht
\graphicspath{{img/}{../../img/}}

\begin{document}

\chapter{Instructions for cutting the cardboard template}\label{cutterchapter}
First of all, you'll need a suitable cardboard to cut out the template. Make sure that the cardboard does not have creases or dent. Ideally, the cardboard has a thickness between $0.5 mm$ and $1.0 mm$ and is not coated. We are recommending a thickness of $1.0 mm$. In addition, it should have a thickness of around $600 g/m^2$. \\
The template is designed for laser cutters, but it's possible to cut it out with a scissor. The thickness of the lines within the template are $0.001 inches$. Certain parts of the original are engraved, they are thicker ($0.01 inches$) than the rest of the original lines. \\
In addition ensure that the settings of your laser cutter have been adjusted properly in regard of thickness. For the recommended thickness of $1.0 mm$ the values for cutting should be around 30 speed, 80 power and 80 frequency. The values for engraving should be around 100 speed, 20 power and 20 frequency.\\
Also ensure that you have set the order of the lines to be cut (from inside to outside). First the green lines are engraved, then the blue lines and finally the outer lines of the cardboard are being cut.\\
For how to reassembling the cardboard, please refer to the reassembling cardboard document.

\end{document}