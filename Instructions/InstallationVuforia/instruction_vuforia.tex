\documentclass[12pt,a4paper,twoside,openright]{scrreprt}

% Sprachpakete
\usepackage[latin1]{inputenc}
\usepackage[T1]{fontenc}
\usepackage{lmodern}
\usepackage[english]{babel}
\usepackage{url}
\usepackage{hyperref}
% Zum plattformunabh�ngigen Einf�gen anderer Tex-Dateien
% Hinweise zur Benutzung weiter unten
\usepackage{subfiles}
% Zum Anpassen der Seitenr�nder
\usepackage{geometry}
% F�r Abk�rzungsverzeichnis und Glossar
\usepackage[nonumberlist,acronym]{glossaries}
% Zum Einbinden von PDF-Seiten
\usepackage{pdfpages}
% Zum Einbinden von Grafiken
\usepackage{graphicx}		
% Um Zellen in Tabellen zusammenzufassen
\usepackage{multirow}
% F�r farbige Schriften und Tabellen
\usepackage{xcolor}
\usepackage{colortbl}
% Package f�r zus�tzliche mathematische Symbole
\usepackage{amssymb}
\usepackage{units} 
% Package zum erstellen von ToDo-Anmerkungen
% Befehl: \todo{Text} oder \todo[inline]{text} f�r Anmerkungen am Rand oder in-line
%					\missingfigure{text} Platzhalter f�r geplante Grafiken/Tabellen mit Beschreibung
% Beim fertigen Dokument muss dieser Package mittels \usepackage[disable]{todonotes} deaktiviert werden,
% um alle vorhandenen ToDos als auch die ToDo-Liste ausgeblendet werden.
%\usepackage{todonotes} 			

% Nummerierung der Grafiken nach Kapitel
\usepackage{chngcntr}
\counterwithin{figure}{section}

% Zum Einbinden von Code-Ausschnitte
% Mittels \lstinputlisting[language=C, firstline=7, lastline=11]{quellcode-datei.c} k�nnen 
% die Zeilen 7 bis 11 der quellcode-datei.c mit der Syntax-Highlighting f�r die Sprache C eingef�gt werden.
% Das Package unterst�tzt eine Vielzahl an Programmiersprachen,
% und am Besten den Code in einer figure-Umgebung setzen
\usepackage{listings}
\usepackage{hyperref}
\lstset{numbers=left, numberstyle=\small, basicstyle=\small, numbersep=3mm, framexleftmargin=8mm, framexrightmargin=0cm, language=C,  frame=single, breaklines=true}

\makeindex

%%%%%%%%%%%%%%%%%%%%%%%%%%
\begin{document}

\pagestyle{plain}

%%%%%%%%%%%%%%%%%%%%%%%%%%%%%%%%%%%%%%%%%%%%%%%%%%%%%%%%%%%%%%%%%%%%%%%%%%%% Titleseite V1
\begin{titlepage}

\begin{center}
\includegraphics[scale=0.3]{LogoUniOL.png}\\[2.2cm]
\Huge{\textbf{Project Group Bull's Eye}}\\[0.4cm]
\huge{\textbf{Instruction: Vuforia}}\\[0.5cm]
\Large{2016~-~2017}\\[3ex]
\large{
Daniela Betzl\\
Tim Lukas Cofala\\
Aljoscha Niazi-Shahabi\\
Stefan Niewerth\\
Henrik Reichmann\\
Rieke von Bargen\\
}
\vspace{1.5cm}
\large{
Prof. Dr. Susanne Boll-Westermann\\
Organization: M.Sc. Marion Koelle \\
Advisors: M.Sc. Tim Stratmann, M.Sc. Uwe Gr�nefeld\\
Last revision: 23.08.2017\\
Version: 1.0\\
Carl von Ossietzky University of Oldenburg\\
\textsc{Faculty II } School of Computing Science, Business Administration, Economics, and Law 
}
\end{center}
\end{titlepage}

% Diese Liste listet alle eingef�gten ToDo-Anmerkungen auf, sodass man ein �berblick �ber allen ToDos im ganzen Dokument hat
%\listoftodos

%%%%%%%%%%%%%%%%%%%%%%%%%%%%%%%%%%%%%%% Verzeichnisse
\pagenumbering{roman}
\tableofcontents \clearpage
%\listoffigures \clearpage
%\listoftables \clearpage
\pagenumbering{arabic}
\clearpage
\setcounter{page}{1}

%%%%%%%%%%%%%%%%%%%%%%%%%%%%%%%%%%%%%%% die Kapitel des Dokuments
% Die einzelne Kapitel k�nnen hier mittels dem Befehl \subfile{chapter/unterordner/textdatei.tex} eingef�gt werden,
% wobei chapter/unterordner/ der relative Pfad zur Datei ist.
% Das Package subfiles sorgt daf�r, dass die einzlenen Kapitel automatisch mit dieser Datei verkn�pft werden, sodass keine 
% Konfiguration des LaTeX-Editors notwendig ist. Dies ist praktisch, da die Konfiguration im Editor manuell erledigt werden muss,
% und von Editor zu Editor sich unterscheiden.
% Zudem erm�glicht subfiles die einzelnen Kapitel direkt zu kompilieren, ohne masterdoc.tex sowie die anderen zugeh�rigen Tex-Dateien
% kompilieren zu m�ssen. Dadurch geht das kompilieren schneller, wobei nachdem man fertig ist masterdoc ebenfalls kompilieren sollte, 
% sodass das Kapitel bzw. die �nderungen am Kapitel im Hauptdokument ebenfalls zu sehen ist.

%%%%%%%%%%%%%%%%%%%%%%%%%%%% Kapitel 1
\subfile{chapter/vuforia/vuforia.tex}

%%%%%%%%%%%%%%%%%%%%%%%%%%%%%%%%%%%%%%% Literaturverzeichnis
% An dieser Stelle wird das Literaturverzeichnis eingef�gt, die Literatur sollte dabei in einer BibTex-Datei vorhanden sein.
% Hierf�r die Zeilen auskommentieren und in der geschweiften Klammer des bibliography-Befehls der Name der BibTex-Datei angeben.
%\bibliography{bibliography}
%\bibliographystyle{lnig}

\end{document}