\documentclass[12pt,a4paper,twoside,openright]{scrreprt}

% Sprachpakete
\usepackage[utf8]{inputenc}
%\usepackage[latin1]{inputenc}
\usepackage[T1]{fontenc}
\usepackage{lmodern}
\usepackage[english]{babel}
% Zum plattformunabhängigen Einfügen anderer Tex-Dateien
% Hinweise zur Benutzung weiter unten
\usepackage{subfiles}
% Zum Anpassen der Seitenränder
\usepackage{geometry}
% Einrücken vermeiden
\usepackage{parskip}
% Zum Einbinden von PDF-Seiten
\usepackage{pdfpages}
% Zum Einbinden von Grafiken
\usepackage{graphicx}		
% Um Zellen in Tabellen zusammenzufassen
\usepackage{multirow}
% Für farbige Schriften und Tabellen
\usepackage{xcolor}
\usepackage{colortbl}
% Package für zusätzliche mathematische Symbole
\usepackage{amssymb}
\usepackage{units} 
\usepackage{hyperref}
% Package zum erstellen von ToDo-Anmerkungen
% Befehl: \todo{Text} oder \todo[inline]{text} für Anmerkungen am Rand oder in-line
%					\missingfigure{text} Platzhalter für geplante Grafiken/Tabellen mit Beschreibung
% Beim fertigen Dokument muss dieser Package mittels \usepackage[disable]{todonotes} deaktiviert werden,
% um alle vorhandenen ToDos als auch die ToDo-Liste ausgeblendet werden.
\usepackage[disable]{todonotes} 			

% Nummerierung der Grafiken nach Kapitel
\usepackage{chngcntr}
\counterwithin{figure}{section}

% Zum Einbinden von Code-Ausschnitte
% Mittels \lstinputlisting[language=C, firstline=7, lastline=11]{quellcode-datei.c} können 
% die Zeilen 7 bis 11 der quellcode-datei.c mit der Syntax-Highlighting für die Sprache C eingefügt werden.
% Das Package unterstützt eine Vielzahl an Programmiersprachen,
% und am Besten den Code in einer figure-Umgebung setzen
\usepackage{listings}
\lstset{numbers=left, numberstyle=\small, basicstyle=\small, numbersep=3mm, framexleftmargin=8mm, framexrightmargin=0cm, language=C,  frame=single, breaklines=true}

%%%%%%%%%%%%%%%%%%%%%%%%%%
\begin{document}

\pagestyle{plain}

%%%%%%%%%%%%%%%%%%%%%%%%%%%%%%%%%%%%%%%%%%%%%%%%%%%%%%%%%%%%%%%%%%%%%%%%%%%% Titleseite V1
\begin{titlepage}

	\begin{center}
	\includegraphics[scale=0.3]{LogoUniOL.png}\\[2.2cm]
	\Huge{\textbf{Project Group Bull's Eye}}\\[0.4cm]
	\huge{\textbf{Installationguide of Unity and Pupil Capture}}\\[0.5cm]
	\Large{2016~-~2017}\\[3ex]
	\large{
		Daniela Betzl\\
		Tim Lukas Cofala\\
		Aljoscha Niazi-Shahabi\\
		Stefan Niewerth\\
		Henrik Reichmann\\
		Rieke von Bargen\\
	}
	\vspace{1.5cm}
	\large{
				Prof. Dr. Susanne Boll-Westermann\\
				Advisors: M.Sc. Tim Stratmann, M.Sc. Uwe Gr\"unefeld\\
				Organization: M.Sc. Marion Koelle \\
				Last revision: 26.09.2017\\
				Version: 2.0\\
				Carl von Ossietzky University of Oldenburg\\
				\textsc{Faculty II } School of Computing Science, Business Administration, Economics, and Law 
			}
	\end{center}
\end{titlepage}

% Diese Liste listet alle eingefügten ToDo-Anmerkungen auf, sodass man ein Überblick über allen ToDos im ganzen Dokument hat
\listoftodos

%%%%%%%%%%%%%%%%%%%%%%%%%%%%%%%%%%%%%%% Abstract
% Auskommentieren, falls kein Abstrakt notwendig ist
%\subfile{chapter/abtract.tex}

%%%%%%%%%%%%%%%%%%%%%%%%%%%%%%%%%%%%%%% Verzeichnisse
\pagenumbering{roman}
\tableofcontents \clearpage
%\listoffigures \clearpage
%\listoftables \clearpage
\pagenumbering{arabic}
\clearpage
\setcounter{page}{1}

%%%%%%%%%%%%%%%%%%%%%%%%%%%%%%%%%%%%%%% die Kapitel des Dokuments
% Die einzelne Kapitel können hier mittels dem Befehl \subfile{chapter/unterordner/textdatei.tex} eingefügt werden,
% wobei chapter/unterordner/ der relative Pfad zur Datei ist.
% Das Package subfiles sorgt dafür, dass die einzlenen Kapitel automatisch mit dieser Datei verknüpft werden, sodass keine 
% Konfiguration des LaTeX-Editors notwendig ist. Dies ist praktisch, da die Konfiguration im Editor manuell erledigt werden muss,
% und von Editor zu Editor sich unterscheiden.
% Zudem ermöglicht subfiles die einzelnen Kapitel direkt zu kompilieren, ohne masterdoc.tex sowie die anderen zugehörigen Tex-Dateien
% kompilieren zu müssen. Dadurch geht das kompilieren schneller, wobei nachdem man fertig ist masterdoc ebenfalls kompilieren sollte, 
% sodass das Kapitel bzw. die Änderungen am Kapitel im Hauptdokument ebenfalls zu sehen ist.

%%%%%%%%%%%%%%%%%%%%%%%%%%%% Kapitel 1
%\addchap{Vorwort}
%Dieses Dokument dient als Zusammenfassung aller Sprints mit den jeweiligen Zeiträumen, Aufgaben, Entscheidungen und Besonderheiten sowie einer kleinen Überblick der jeweiligen Sprints. Zudem fungiert es als Nachschlagewerk.
\subfile{chapter/unity/unity.tex}
\subfile{chapter/pupil/pupil.tex}

%%%%%%%%%%%%%%%%%%%%%%%%%%%%%%%%%%%%%%% Literaturverzeichnis
% An dieser Stelle wird das Literaturverzeichnis eingefügt, die Literatur sollte dabei in einer BibTex-Datei vorhanden sein.
% Hierfür die Zeilen auskommentieren und in der geschweiften Klammer des bibliography-Befehls der Name der BibTex-Datei angeben.
%\bibliography{bibliography}
%\bibliographystyle{lnig}

\end{document}